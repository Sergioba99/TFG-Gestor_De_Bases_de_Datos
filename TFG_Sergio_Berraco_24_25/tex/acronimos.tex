% Definición de acrónimos:
% \newacronym{id}{corto}{largo}

% Uso de acrónimos
% \acrshort{id} nombre corto
% \acrlong{id}  nombre largo
% \acrfull{id}  nombre largo (nombre corto)

\newacronym{TFG}{TFG}{Trabajo Fin de Grado}
\newacronym{CD}{CD}{Compact Disc}
\newacronym{GNU}{GNU}{GNU is Not Unix}
\newacronym{PDF}{PDF}{Portable Document Format}
\newacronym{TCPIP}{TCP/IP}{Pila de protocolos de Internet}
\newacronym{TCP}{TCP}{Transport Control Protocol}
\newacronym{XML}{XML}{eXtensible Markup Language}
\newacronym{SQL}{SQL}{Structured Query Languaje}
\newacronym{TB}{TB}{Tera Byte}
\newacronym{GB}{GB}{Giga Byte}
\newacronym{MB}{MB}{Mega Byte}
\newacronym{KB}{KB}{Kilo Byte}
\newacronym{EDR}{EDR}{Esquema de diseño relacional}
\newacronym{ROBIN}{ROBIN}{Rail mOBIlity simulatioN}
\newacronym{JSON}{JSON}{JavaScript Object Notation}
\newacronym{Yaml}{Yaml}{YAML Ain't Markup Language}
\newacronym{HTML}{HTML}{HyperText Markup Language}
\newacronym{CSV}{CSV}{Comma-Separated Values}
\newacronym{API}{API}{Application Programming Interface}
\newacronym{GDI}{GDI}{Graphics Device Interface}
\newacronym{Tcl}{Tcl}{Tool Command Language}