\begin{resumen}


%Los grupos \textit{ORETO} y \textit{MAT} de la Escuela Superior de Informática de la Universidad de Castilla-La Mancha han desarrollado \acrfull{ROBIN}~\cite{delCastilloHerrera2024ROBIN}, un simulador de movilidad ferroviaria microscópico que permite modelar el flujo de viajeros en corredores ferroviarios. Este simulador emplea varios ficheros como entrada de datos y, al acabar la simulación, genera unos archivos donde se guardan los resultados de cada simulación. El manejo de estos ficheros es tedioso en ocasiones debido a la cantidad de ficheros que se deben gestionar y al tamaño de los mismos. Por esto, se ha decidido la utilización de una base de datos para estos propósitos. 

%El objetivo de este \acrlong{TFG} consiste en el diseño e implementación de un conjunto de bases de datos destinadas a almacenar los distintos archivos de configuración para la oferta y la demanda del simulador de movilidad ferroviaria \acrfull{ROBIN}, así como los archivos de resultados que éste genere. Además, se ha propuesto desarrollar una interfaz simple que facilite la gestión de dichas bases de datos, permitiendo la importación, regeneración y consulta de los archivos a través de sentencias \acrfull{SQL}.

%Para diseñar las bases de datos necesarias para la consecución de este TFG, primero se ha estudiado el funcionamiento y diseño de las bases de datos relacionales y los diferentes comandos empleados en el lenguaje SQL para realizar sentencias que permitan crear las bases de datos con sus correspondientes tablas, editar las tablas de las bases de datos para añadir o eliminar información contenida en las mismas y realizar consultas para visualizar los datos almacenados en dichas tablas de las bases de datos. En la fase de diseño se ha empleado la herramienta SQLiteStudio, que utiliza SQLite3 como motor para las bases de datos y permite generar bases de datos, con sus correspondientes tablas, e interactuar con ellas por medio de su interfaz o bien mediante sentencias SQL.

%A continuación, se ha abordado la creación de los diferentes módulos que formarán parte del programa. Cada módulo tiene una función diferente, por ejemplo, hay un módulo encargado de leer y procesar la información de los archivos de configuración, tanto los que se usan para configurar la oferta como los que se emplean en configurar la demanda. También hay otro módulo encargado de la gestión de las bases de datos, es decir, tiene como cometido crear las bases de datos con sus diferentes tablas, insertar los datos provenientes de los diferentes archivos, borrar la información de los archivos que se eliminen de las bases de datos o ejecutar las sentencias SQL en la base de datos para, por ejemplo, consultar datos de un archivo concreto. \textbf{TEMA: SE NOMBRAN TODOS? NO ME PARECE MAL, AUNQUE NO SÉ SI DEMASIADO DETALLE}

%Una vez verificado el correcto funcionamiento de cada uno de los módulos que componen el programa, se ha diseñado e implementado una interfaz que, por medio de los módulos mencionados, gestiona las bases de datos. Esta interfaz tiene la capacidad de importar los datos de los archivos a la base de datos, regenerar estos archivos de nuevo para su uso en el simulador y ejecutar sentencias SQL para consultar datos de las bases de datos. Estas consultas pueden almacenarse en el programa para su uso en ocasiones posteriores, también se pueden cargar archivos \texttt{.sql} que contengan la sentencia para ejecutarla con la aplicación o escribir las sentencias directamente en la entrada de texto que posee la aplicación.


Los grupos \textit{ORETO} y \textit{MAT} de la Escuela Superior de Informática de la Universidad de Castilla-La Mancha han desarrollado \acrfull{ROBIN}~\cite{delCastilloHerrera2024ROBIN}, un simulador de movilidad ferroviaria microscópico que permite modelar el flujo de viajeros en corredores ferroviarios. Este simulador emplea varios ficheros como entrada de datos y, al acabar la simulación, genera unos archivos donde se guardan los resultados de cada simulación. El manejo de estos ficheros es tedioso en ocasiones debido a la cantidad de ficheros que se deben gestionar. Por esto, se ha decidido la utilización de una base de datos para estos propósitos.

El objetivo de este \acrlong{TFG} consiste en el diseño e implementación de un conjunto de bases de datos destinadas a almacenar los distintos archivos de entrada de datos para la oferta y la demanda del simulador de movilidad ferroviaria \acrshort{ROBIN}, así como los archivos de resultados generados. Además, se propone desarrollar una interfaz gráfica que facilite la gestión de dichas bases de datos, permitiendo la importación, regeneración y consulta de los archivos a través de sentencias \acrfull{SQL}.

Para diseñar las bases de datos necesarias para la consecución de este TFG, primero se ha estudiado el funcionamiento y diseño de las bases de datos relacionales. A continuación, se han analizado los diferentes comandos empleados en el lenguaje SQL para crear las bases de datos con sus correspondientes tablas, editar estas tablas para añadir o eliminar información y ejecutar consultas que permitan visualizar los datos almacenados en ellas.

Una vez entendido el funcionamiento de las bases de datos relacionales, se ha pasado a estudiar los diferentes bloques que poseen los archivos \acrshort{Yaml} y \acrshort{CSV} para poder diseñar de las bases de datos necesarias que contengan toda la información que estos archivos tienen almacenada. En la fase de diseño se ha empleado la herramienta SQLiteStudio, que utiliza SQLite3 como motor para las bases de datos. SQLiteStudio permite generar bases de datos, con sus correspondientes tablas, e interactuar con ellas por medio de su interfaz o bien mediante sentencias SQL.

Después, se ha abordado la creación del diseño final de aplicación que manejará todo. Se ha diseñado una estructura modular, donde cada módulo tiene una función diferente, por ejemplo, hay un módulo encargado de leer y procesar la información de los archivos de entrada de datos para \acrshort{ROBIN}, también hay otro módulo encargado de la gestión de las bases de datos y la ejecución de sentencias \acrshort{SQL} en las bases de datos. Se ha comprobado el correcto funcionamiento de estos módulos.

Finalmente, se ha diseñado e implementado una interfaz que, utilizando los módulos que componen el programa, gestiona las bases de datos. Esta interfaz tiene la capacidad de importar los datos de los archivos a la base de datos, regenerar estos archivos de nuevo para su uso en el simulador y ejecutar sentencias SQL para consultar datos de las bases de datos. Estas consultas pueden almacenarse en el programa para su uso en ocasiones posteriores, también se pueden cargar archivos \texttt{.sql} que contengan la sentencia para ejecutarla con la aplicación o escribir las sentencias directamente en la entrada de texto que posee la aplicación.


\end{resumen}

\begin{abstract}

The \textit{ORETO} and \textit{MAT} research groups at the Higher School of Computer Science of the University of Castilla–-La Mancha have developed \acrfull{ROBIN}~\cite{delCastilloHerrera2024ROBIN}, a microscopic railway mobility simulator that allows modeling passenger flows in railway corridors. This simulator uses several files as input data and, upon completion of the simulation, generates files in which the results of each run are stored. Managing these files can be tedious at times due to the number of files that must be handled. Therefore, a database has been adopted for these purposes.

The objective of this Final Degree Project (TFG) is the design and implementation of a set of databases intended to store the various input data files for the supply and demand of \acrshort{ROBIN}, as well as the generated result files. Additionally, a graphical user interface is proposed to facilitate the management of these databases, enabling the import, rebuilding, and querying of the files via \acrfull{SQL} statements.

To design the databases necessary for the completion of this \acrshort{TFG}, the operation and design of relational databases have first been studied. Subsequently, the various commands employed in the \acrfull{SQL} language for creating databases with their corresponding tables, editing these tables to add or remove information, and performing queries to display the data stored within have been examined.

Once the operation of relational databases was understood, the different blocks present in \acrshort{Yaml} and \acrshort{CSV} files were analyzed to design the necessary databases that would contain all the information stored within these files. In the design phase, the SQLiteStudio tool was used, which employs SQLite3 as the database engine. SQLiteStudio allows creating databases with their corresponding tables and interacting with them either through its interface or via SQL statements.

Subsequently, the creation of the final application design that will manage everything has been undertaken. A modular structure has been designed, in which each module has a different function; for example, one module is responsible for reading and processing the input data files for \acrshort{ROBIN}, and another module is responsible for database management and the execution of \acrshort{SQL} statements. The correct operation of these modules has been verified.

Finally, an interface has been designed and implemented that, using the modules comprising the program, manages the databases. This interface is capable of importing data from the files into the database, reconstructing these files for use in the simulator, and executing \acrshort{SQL} statements to query the databases. These queries can be stored within the application for later use; additionally, \texttt{.sql} files containing statements can be loaded for execution, or statements can be entered directly into the application's text input.

\end{abstract}
