\chapter{Objetivos}
\label{ch:objetivos}

El objetivo principal de este \acrfull{TFG} consiste en el diseño e implementación de una serie de bases de datos destinadas a almacenar los diferentes datos empleados en la simulación de transporte ferroviario realizada por \acrshort{ROBIN}, así como los resultados de las simulaciones que se realicen. Para ello, se utilizará el gestor de bases de datos relacionales SQLite3. Además, se diseñará e implementará una interfaz para interactuar con las diferentes bases de datos, con la ayuda de Python y la librería para generar interfaces Tkinter. Para la consecución de este objetivo se deben realizar los siguientes objetivos parciales:

\begin{itemize}

\item Estudiar el funcionamiento y diseño de las bases de datos relacionales y su aplicación concreta dentro de SQLite3.

\item Diseñar e implementar las diferentes bases de datos.

\item Diseñar y desarrollar un programa en Python que permita interactuar con las bases de datos, incluyendo la inserción, exportación, eliminación de datos y ejecución de sentencias SQL. El diseño será modular para poder añadir nuevas funcionalidades en el futuro.

\item Investigar el proceso de creación de interfaces gráficas con Tkinter.

\item Diseñar e implementar una interfaz gráfica para facilitar el uso del programa desarrollado.
\end{itemize}