\chapter{Introducción} 
\label{ch:introduccion}
A día de hoy, en el transporte ferroviario contemporáneo, alcanzar un equilibrio entre los asientos ofertados por las compañías que ofrecen servicios ferroviarios y la demanda real de pasajeros constituye un gran reto operativo. Esto se debe a la liberación y al aumento de la competencia en el sector ferroviario, sobre todo en el europeo, donde múltiples operadores luchan por captar a los mismos pasajeros. La venta de billetes ha de ajustarse a la capacidad efectiva de cada tren para así evitar tanto la saturación de dichos servicios como el desaprovechamiento de plazas, lo que produciría una bajada en el rendimiento del servicio.

La liberalización progresiva del sector ferroviario viene impulsada por la Unión Europea desde los años noventa, lo que abre el mercado a múltiples operadores. Se ha materializado en cuatro paquetes legislativos (1991, 2001, 2012 y 2016)\footnote{Recopilación de los diferentes paquetes legislativos del sector ferroviario por la Comisión Europea: \url{https://transport.ec.europa.eu/transport-modes/rail/railway-packages_en} (Último acceso: 8 de julio de 2025)} que establecen la separación de la infraestructura de la operación de servicios, la creación de organismos reguladores independientes y la armonización técnica para fomentar un espacio ferroviario europeo único. Estos marcos legales han permitido la entrada de nuevos competidores en corredores estratégicos, modificando profundamente la forma de asignar capacidad y gestionar la demanda.

En el caso concreto de España, tras su adhesión a la Unión Europea en 1986\footnote{Ley Orgánica 10/1985, de 2 de agosto, de Autorización para la Adhesión de España a las Comunidades Europeas, publicado en BOE num. 189, de 8/8/1985, páginas 25119 a 25119 (1 pág.), referencia BOE-A-1985-16659 \url{https://www.boe.es/eli/es/lo/1985/08/02/10} (Último acceso: 8 de julio de 2025)}, la transposición de la política comunitaria se inició con la Ley 39/2003\footnote{Ley 39/2003, de 17 de noviembre, del Sector Ferroviario, publicado en BOE num. 276, de 18/11/2003 con referencia BOE-A-2003-20978 \url{https://www.boe.es/eli/es/l/2003/11/17/39/con} (Último acceso: 8 de julio de 2025)}, que rompió el antiguo monopolio de RENFE, y separó las competencias de la empresa en dos: gestión de la infraestructura (ADIF\footnote{Sitio web de presentación de Adif: \url{https://www.adif.es/sobre-adif/conoce-adif/quienes-somos} (Último acceso: 8 de julio de 2025)}) y explotación (Grupo Renfe\footnote{Sitio web de presentación de \textit{Grupo Renfe} y sus diferentes sociedades: \url{https://www.renfe.com/es/es/grupo-renfe/grupo-renfe} (Último acceso: 8 de julio de 2025)}). Posteriormente, la Ley 38/2015\footnote{Ley 38/2015, de 29 de septiembre, del sector ferroviario, publicado en BOE num. 234, de 30/09/2015 con referencia BOE-A-2015-10440 \url{https://www.boe.es/eli/es/l/2015/09/29/38/con} (Último acceso: 8 de julio de 2025)} y el Real Decreto-ley 23/2018\footnote{Real Decreto-ley 23/2018, de 21 de diciembre, de transposición de directivas en materia de marcas, transporte ferroviario y viajes combinados y servicios de viaje vinculados, publicado en BOE núm. 312, de 27/12/2018 con referencia BOE-A-2018-17769 \url{https://www.boe.es/eli/es/rdl/2018/12/21/23} (Último acceso: 8 de julio de 2025)} incorporaron íntegramente los paquetes ferroviarios europeos, fijando diciembre de 2020 como fecha de inicio efectiva de la liberalización del transporte de pasajeros en alta velocidad y larga distancia. En noviembre de 2019, ADIF adjudicó tres paquetes de capacidad (65\%, 30\% y 5\%) para los corredores Madrid -- Barcelona -- Frontera Francesa, Madrid -- Levante y Madrid -- Toledo -- Sevilla -- Málaga, reservándolos a Renfe, Iryo y Ouigo respectivamente. Estos operadores comenzaron sus servicios entre 2022 (Ouigo y AVLO, el servicio de bajo coste de Renfe) y diciembre de 2022 (Iryo), consolidando así la competencia en los principales corredores de alta velocidad.

La liberalización de cualquier mercado en general (y del ferroviario en particular) motiva la entrada de nuevas empresas que competirán por la demanda de los usuarios. En el caso del sector ferroviario, la infraestructura juega un papel fundamental y limitante a la hora de coordinar las peticiones de uso de la misma por parte de las compañías. Lo anterior supone un reto para el gestor de la infraestructura que debe organizar el uso de la infraestructura entre diferentes operadores considerando para ello tanto indicadores económicos como de equidad entre empresas. Por otra parte, las propias compañías deben enfrentar otra serie de desafíos motivados, entre otros, por la variabilidad en la demanda, lo que repercute, por ejemplo, en los servicios que se pondrán a disposición de la demanda así como sus precios.

Por este motivo, en la convocatoria nacional del Ministerio de Ciencia e Innovación de 2020, se solicitó, y fue concedido, el proyecto \textit{Técnicas Analíticas para la transición hacia un transporte sostenible de pasajeros y mercancías en entornos de competitividad (ACoSeM@SusTran)}. Es un proyecto coordinado por las Universidades Universidad Politécnica de Cataluña, la Universidad Rey Juan Carlos y la Universidad de Castilla-La Mancha. Esta última ejecuta el proyecto titulado \textit{Meeting Artificial Intelligence and Machine Learning for Rail Passenger Service Planning under Competition (PID2020-112967GB-C32)}. El objetivo principal de este proyecto consiste en el diseño de métodos de aprendizaje automático e inteligencia artificial para el análisis táctico de la competencia en los sistemas de transporte. Se pretende abordar los objetivos utilizando métodos de aprendizaje automático e inteligencia artificial que se basan en datos y no dependen del modelizador. La obtención de datos para utilizarlos en las investigaciones es un proceso complejo, dado que las compañías no ceden su información, y, en muchos casos, no existen. Por todo esto, se hace muy necesario el uso de herramientas de simulación robustas con este fin. Por este motivo, desde los grupos ORETO y MAT de la Escuela Superior de Informática de la Universidad de Castilla-La Mancha se ha desarrollado \acrfull{ROBIN}~\cite{delCastilloHerrera2024ROBIN}. \acrshort{ROBIN}, un simulador de movilidad ferroviaria microscópico que permite modelar el flujo de viajeros en corredores ferroviarios. Este simulador emplea varios ficheros como entrada de datos y, al acabar la simulación, genera unos archivos donde se guardan los resultados de la misma. El manejo de estos archivos puede ser tedioso, debido a la cantidad de datos que se deben gestionar. Esto puede resultar un problema si se quieren tener todos los datos almacenados de manera ordenada en un mismo lugar. 

Por ello, el objetivo de este \acrshort{TFG} consiste en diseñar e implementar una serie de bases de datos encargadas de almacenar los ficheros que emplea \acrshort{ROBIN} como entrada de información de la oferta y de la demanda, así como los archivos de resultados que se generan después de las simulaciones. Las bases de datos irán acompañadas de un programa destinado a su gestión por medio de una interfaz. Dicho programa tendrá la capacidad de importar los archivos a las bases de datos, exportarlos de las bases de datos, y eliminar los archivos que no sean útiles de dichas bases de datos. Esto permitirá ahorrar tiempo en la gestión de los archivos, ya que todos se encontrarán centralizados en un mismo punto.

\section{Organización de la memoria} 
\label{sec:organizacion-memoria}

Este documento se ha estructurado en los siguientes capítulos:
\begin{description}
    \item[\autoref{ch:objetivos}] Se definirá el objetivo principal del \acrshort{TFG}, así como los diferentes objetivos que se deben llevar a cabo para completar el objetivo principal.
    \item[\autoref{ch:antecedentes}] Se analizan los conocimientos previos y las tecnologías utilizadas por \acrshort{ROBIN} y los problemas que plantea. Además, se exponen las posibles soluciones que podrían utilizarse para mejorar los problemas planteados, así como las tecnologías que podrían emplearse con este fin.
    \item[\autoref{ch:desarrollo}] Se describe el proceso completo de diseño e implementación del proyecto. Este capítulo se ha dividido en 4 secciones:
    \item[\autoref{sec:estudioBasesDeDatos}] Se exploran tanto el diseño como el funcionamiento de las bases de datos relacionales. 
    \item[\autoref{sec:archivosEntradaSalida}] Se expone la estructura de los diferentes archivos empleados por \acrshort{ROBIN} para la entrada y salida de datos.
    \item[\autoref{sec:diseñoImplementacionBasesDeDatos}] Se detalla el diseño e implementación de las bases de datos necesarias para la consecución del proyecto.
    \item[\autoref{sec:desarrolloSoftware}] Se muestra el diseño del software encargado de gestionar las bases de datos, así como de la creación de una interfaz para facilitar su uso.
    \item[\autoref{ch:resultados}] Se analizan los resultados del proyecto obtenidos. Este capítulo se ha dividido en dos secciones.
    \item[\autoref{sec:UIResults}] Se realizarán pruebas para verificar el correcto funcionamiento del programa y verificar que es capaz de capturar ciertos errores que puedan surgir durante su uso.
    \item[\autoref{sec:compareFilesResults}] Se comprueba que las bases de datos funcionan correctamente, es decir, que no corrompan o modifiquen los datos de los diferentes archivos almacenados en estas. Esto se ha hecho comparando los archivos originales con los que se exportan utilizando la aplicación.    
    \item[\autoref{ch:conclusiones}] Se recopilan las conclusiones del proyecto, los objetivos propuestos que se han cumplido, los objetivos académicos logrados durante el desarrollo del proyecto y se plantean posibles líneas de trabajo futuro.
    \item[\deschyperlink{ch:anexos}{Anexos}] Se encuentran los esquemas de diseño relacional para cada una de las bases de datos, las plantillas estructurales de los archivos gestionados y una introducción a la lógica difusa.
    \item[\deschyperlink{ch:bibliografia}{Bibliografía}] Recoge las referencias y recursos técnicos utilizados para el desarrollo y redacción de este \acrshort{TFG}.
\end{description}

\section{Repositorio de información}
\label{sec:repositorio}

%\warning{Es muy útil tanto para la ejecución como para la evaluación disponer de un repositorio para almacenar las sucesivas versiones del documento y de todo el material generado durante el proyecto.  La UCLM incorpora \href{https://github.com}{GitHub} como servicio institucional. Si no utilizas un repositorio quita esta sección.}

Todo el material generado durante la ejecución de este proyecto está disponible en el repositorio \thegitrepo{}.  El material incluye el código \LaTeX{} del presente documento, el código fuente de los programas realizados o modificados, y todos los datos generados en la evaluación de resultados.